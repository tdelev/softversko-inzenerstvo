\documentclass[12pt,a4paper]{exam}
\usepackage{amsmath}
\usepackage{amsfonts}
\usepackage{amssymb}
\usepackage{ucs}
\usepackage[T2A]{fontenc}
\usepackage[utf8]{inputenc}
\usepackage[english,bulgarian]{babel}
\usepackage{listings}
\usepackage{color}
\definecolor{lightgrey}{rgb}{0.9,0.9,0.9}
\usepackage[usenames,dvipsnames]{xcolor}
\lstset{language=C,captionpos=b,
tabsize=4,frame=lines,
basicstyle=\tiny\ttfamily,
keywordstyle=\color{blue},
commentstyle=\color{lightgrey},
stringstyle=\color{violet},
breaklines=true,showstringspaces=false}


\begin{document}
\pagestyle{headandfoot}
\header{\textbf{ФИНКИ/ФЕИТ\\Софтверско
инженерство}}{}{\large{\textbf{Лабораториска вежба 1}}}
\headrule
\cfoot{Страна \thepage}
\begin{center}
\Large{\textbf{Шаблони и исклучоци}}
\end{center}
\begin{questions}

\question
Да се напише генеричка (шаблон) функција за наоѓање максималниот елемент во
низа. Функцијата треба да работи за низи од цели броеви (\texttt{int}), децимални броеви
(\texttt{float, double}) и знаци (\texttt{char}).

\question
Да се напише шаблон (темплејт) класа за нумеричка листа кој ќе работи само со нумерички
типови. Класата треба да ги овозможува следните функционалности: 
\begin{itemize}
  \item додавање нов елемент на почеток и крај на листата
  \item бришење на елемент на почеток и крај на листата.  
\end{itemize}
Дополнително да се имплементираат следните методи: 
\begin{itemize}
  \item метод кој ќе ја пресметува средната вредност ($\mu$) на елементите во
  низата (се извршува во константно време О(1))
  \item метод кој ќе ја пресметува стандардната девијација ($\sigma$) на
  елементите во низата (се пресметува со следната формула: $\sigma =
  \sqrt{\frac{1}{N} \sum_{i=1}^N (x_i - \mu)^2}, \mu =
  \frac{1}{N} \sum_{i=1}^N x_i$
  \item метод кој ќе ги печати информациите за листата во следниот облик:
  \begin{verbatim}
  Elements: 5 12 7 9 31
  Mean: 12.8
  Std dev: 9.38  
  \end{verbatim}
  \item метод кој ќе пребарува во листата дали постои одреден елемент и ќе ја
  враќа неговата позиција, ако не постои елементот треба да фрли исклучок од тип
  \texttt{ElementNotFoundException}
\end{itemize}

Да се напише главна програма во која ќе се демонстрира употребата на класата:
\begin{enumerate}
  \item се инстанцира објект од класата нумеричка листа;
  \item се пополнува со 5 (пет) елементи;
  \item се повикува методот за печатење;
  \item се пребарува елемент кој не постои (за да се фрли исклучокот и да се
  справиме со него)
  \item се бришат два елементи и повторно се повикува методот за печатење.
\end{enumerate}

\end{questions}
\end{document}